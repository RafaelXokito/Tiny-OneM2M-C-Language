%% 
%% Copyright 2019-2020 Elsevier Ltd
%% 
%% This file is part of the 'CAS Bundle'.
%% --------------------------------------
%% 
%% It may be distributed under the conditions of the LaTeX Project Public
%% License, either version 1.2 of this license or (at your option) any
%% later version.  The latest version of this license is in
%%    http://www.latex-project.org/lppl.txt
%% and version 1.2 or later is part of all distributions of LaTeX
%% version 1999/12/01 or later.
%% 
%% The list of all files belonging to the 'CAS Bundle' is
%% given in the file `manifest.txt'.
%% 
%% Template article for cas-dc documentclass for 
%% double column output.

%\documentclass[a4paper,fleqn,longmktitle]{cas-dc}
\documentclass[a4paper,fleqn]{cas-dc}

%\usepackage[authoryear,longnamesfirst]{natbib}
%\usepackage[authoryear]{natbib}
\usepackage[numbers]{natbib}

%%%Author definitions
\def\tsc#1{\csdef{#1}{\textsc{\lowercase{#1}}\xspace}}
\tsc{WGM}
\tsc{QE}
\tsc{EP}
\tsc{PMS}
\tsc{BEC}
\tsc{DE}
%%%
% -------------------------------------------------------------------
% Pacotes para inserção de figuras e subfiguras
\usepackage{subfig,epsfig,tikz,float}		            % Packages de figuras. 
\usepackage{graphicx}
\graphicspath{ {./figs/} }
% -------------------------------------------------------------------
% \usepackage{amssymb}
% -------------------------------------------------------------------
% Pacotes para inserção de tabelas
\usepackage{booktabs,multicol,multirow,tabularx,array}          % Packages para tabela
\usepackage{natbib}
\usepackage{pifont}
\usepackage{xcolor}
% -------------------------------------------------------------------
\PassOptionsToPackage{style=super,nolist}{glossaries}
\PassOptionsToPackage{acronym}{glossaries}
\PassOptionsToPackage{nonumberlist}{glossaries}
\usepackage{glossaries}
\newacronym{ai}{AI}{Artificial Intelligence}
\makeglossaries
% -------------------------------------------------------------------
\usepackage[utf8]{inputenc} % The default since 2018
\DeclareUnicodeCharacter{200B}{{\hskip 0pt}}
% -------------------------------------------------------------------
\begin{document}
\let\WriteBookmarks\relax
\def\floatpagepagefraction{1}
\def\textpagefraction{.001}
\shorttitle{Leveraging social media news}
\shortauthors{CV Radhakrishnan et~al.}

\title [mode = title]{Tiny oneM2M C Language API}                      

\credit{Conceptualization of this study, Methodology, Software}

\author[1]{Rafael Mendes Pereira}[type=editor,
                        %auid=000,bioid=1,
                        linkedin='rafaelmendespereira',
                        orcid=0000-0001-8313-7253]
%\cormark[1]
%\fnmark[1]
\ead{rafael.m.pereira@ipleiria.pt}
   
\address[1]{Computer Science and Communications Research Centre, School of Technology and Management, Polytechnic of Leiria, 2411-901 Leiria, Portugal}

\begin{abstract}
This template helps you to create a properly formatted \LaTeX\ manuscript.

\noindent\texttt{\textbackslash begin{abstract}} \dots 
\texttt{\textbackslash end{abstract}} and
\verb+\begin{keyword}+ \verb+...+ \verb+\end{keyword}+ 
which
contain the abstract and keywords respectively. 

\noindent Each keyword shall be separated by a \verb+\sep+ command.
\end{abstract}

%\begin{graphicalabstract}
%\includegraphics{figs/grabs.pdf}
%\end{graphicalabstract}
%
%\begin{highlights}
%\item Research highlights item 1
%\item Research highlights item 2
%\item Research highlights item 3
%\end{highlights}

\begin{keywords}
quadrupole exciton \sep polariton \sep \WGM \sep \BEC
\end{keywords}


\maketitle

\section{Introdution}

OneM2M is a standards organization that was established in 2012 to provide a common platform for Machine-to-Machine (M2M) and Internet of Things (IoT) communication. M2M refers to the direct communication between devices or machines, without the need for human intervention. This type of communication enables the seamless transfer of data and information, which is critical for various applications, including remote monitoring, control, and automation.

Machine Type Communication (MTC) is a type of M2M communication that enables devices to communicate with each other over wireless networks. MTC is particularly important in the context of the IoT, where a large number of devices need to communicate with each other and with centralized servers to enable various applications.

MTC can be used in various domains, including healthcare, transportation, and industrial automation. However, MTC communication poses unique challenges, including the need for low-power, low-cost devices that can operate reliably in harsh environments. To address these challenges, oneM2M has developed a set of global standards for M2M and IoT communication, which enable seamless interoperability between devices and applications from different vendors and across different networks. These standards provide a common framework for M2M and IoT communication, which helps to ensure that devices and applications can communicate with each other efficiently and securely, enabling the full potential of the IoT to be realized.

The C programming language has been widely used in the development of various software applications, including those related to Machine-to-Machine (M2M) communication and the Internet of Things (IoT). oneM2M and Machine Type Communication (MTC) are two important aspects of M2M and IoT communication that provide a common platform for devices and applications to communicate with each other in a seamless and standardized way.

By adopting oneM2M standards and leveraging the power of C language, developers can create efficient and reliable applications for M2M and IoT communication. C language is known for its high performance, low-level access to memory, and portability across different platforms and architectures, which makes it an ideal choice for developing M2M and IoT applications.

Furthermore, MTC offers unique features and capabilities that are tailored to the needs of the IoT, such as low-power, low-cost devices, and reliable communication in harsh environments. By using C language and oneM2M standards, developers can take advantage of these features to create innovative and impactful applications that solve real-world problems in various domains, including healthcare, transportation, and industrial automation.

The main objectives are to simplify the development of M2M and IoT applications, promote interoperability between devices and applications, and enhance the overall performance and efficiency of communication. The use of a standardized API helps avoid vendor lock-in and ensures compatibility with different devices and platforms. C language offers advantages like high performance, low-level access to memory, and portability. Overall, the creation of a oneM2M API using C language is an important step towards promoting interoperability and simplifying M2M and IoT application development.

\textcolor{red}{Paper structure}

\section{Related work}

% https://www.scopus.com/results/results.uri?sort=plf-f&src=s&st1=%28onem2m+OR+mtc%29+and+%22API%22&sid=2e6f61fce202fa9e6635e553544fee3c&sot=b&sdt=b&sl=40&s=TITLE-ABS-KEY%28%28onem2m+OR+mtc%29+and+%22API%22%29&origin=searchbasic&editSaveS=&yearFrom=Before+1960&yearTo=Present

\printcredits

%% Loading bibliography style file
%\bibliographystyle{model1-num-names}
%\bibliographystyle{cas-model2-names}
\bibliographystyle{unsrt} % Estilo de Bibliografia
% Loading bibliography database
\bibliography{cas-refs}


%\vskip3pt

\bio{figs/bioFig}
My name is Rafael Pereira, I'm 22 years old, and I'm an MSc Computer Engineering Student and Researcher at Polytechnic of Leiria. I'm extremely curious about the technology world, ambitious for knowledge in this area, I'm dedicated to doing my work. The programming thing always got me, and every day it grows. Today I'm looking for interesting projects, that can make me think and learn every day.
\endbio

\end{document}
